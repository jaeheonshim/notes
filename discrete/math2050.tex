\documentclass[12pt]{article}
\usepackage[margin=0.75in,tmargin=1in]{geometry}
\usepackage{fancyhdr}
\usepackage{amsmath}
\usepackage{amsthm}
\usepackage{amssymb}

\title{MATH 3012: Applied Combinatorics}
\author{Jaeheon Shim}
\date{}

\begin{document}

\pagestyle{fancy}
\fancyhead[L]{MATH 3012: Applied Combinatorics}

\maketitle
\thispagestyle{fancy}

\section{Proofs}
\subsection{Axioms}
An argument in a proof is either an axiom or rests on an axiom. An axiom is an unproven statement, and different theorems can becom true or false depending on your choice of axioms.\\\\
The following axioms/assumptions are allowed
\begin{itemize}
	\item The rules of algebra \\
		e.g. if x, y, z are real numbers and $x = y$, then $x + z = y + z$
	\item The set of integers is closed under addition, multiplication, and subtraction
	\item Every integer is either even or odd
	\item If x is an integer, there is no integer between x and x + 1
	\item The relative order of any two real numbers \\
		e.g. $\frac{1}{2} < 1$ or $4.2 \geq 3.7$
	\item The square of any number is greater than or equal to 0
\end{itemize}
\subsection{Existential Instantiation}
A law of logic that says if an object is known to exist, that object can be given a name as long as the name is not currently being used to denote something else. \\\\
\textbf{Example}\\
If $n$ is an odd integer, $n = 2k + 1$ for some integer $k$.
\subsection{Direct Proofs}
In a direct proof of a conditional statement $p \rightarrow c$, the hypothesis $p$ is assumed to be true and the conclusion $c$ is proven as a direct result of the assumption.
\begin{center}
	If n is an odd integer, then $n^2$ is an odd integer.
\end{center}
Many theorems also have a universal quantifier such as
\begin{center}
	For every integer n, if n is odd then $n^2$ is odd.
\end{center}

\subsubsection{Example}
\textbf{Theorem} \quad The square of every odd integer is also odd\\
\begin{proof}
	Let $n$ be an odd integer.\\
	Since $n$ is odd, $n = 2k + 1$ for some integer k. \\
	Plug $n = 2 + 1$ into $n^2$ to get: \\
	\begin{align}
		n^2 &= (2k + 1)^2 \\
			&= 4k^2 + 4k + 1 \\
			&= 2(2^2 + 2k) + 1
	\end{align}
	Since k is an integer, $2^2 + 2$ is also an integer. \\
	Since $n^2 = 2m + 1$, where $m = 2k^2 + 2$ is an integer, $n^2$ is odd.
\end{proof}

\textbf{Two-Column Proof Format} \\
\begin{tabular}{c|c}
	n is odd & Assume p. \\
	$n = 2k + 1$, for some $k \in \mathbf{Z}$ & Definition of Odd \\
	$n^2 = (2k + 1)^2$ & Square both sides \\
	$n^2 = 4k^2 + 4k + 1$ & expand $(2k+1)^2$ \\
	$n^2 = 2(2k^2 + 2k) + 1$ & factor out 2 \\
	$w = 2k^2 + 2m$ & define new variable \\
	$w$ is an integer & integers are closed under addition, multiplication, exponentiation \\
	$n^2 + 2w + 1$, w is integer & substitute w \\
	$n^2$ is odd & definition of odd \\
\end{tabular}
$\therefore$ Therefore, by direct proof, I have shown $p \rightarrow q$
\subsubsection{Example}
Prove that, if x and y are squares, then xy is a square \\\\
p: x and y are squares\\
q: xy is a square \\
prove $p \rightarrow q$
\begin{align*}
	\text{x and y are square integers} && \text{assume p} \\
	x = k^2, k \in \mathbf{Z} && \text{definition of square} \\
	y = j^2, j \in \mathbf{Z} && \text{definition of square} \\
	xy = k^2n^2 && \text{multiply two equations} \\
	xy = g^2, g \in \mathbf{Z}
\end{align*}
Therefore, by direct proof, I have shown $p \rightarrow q$
\end{document}
