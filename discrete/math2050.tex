\documentclass[12pt]{article}
\usepackage[margin=0.75in,tmargin=1in]{geometry}
\usepackage{fancyhdr}
\usepackage{amsmath}
\usepackage{amsthm}
\usepackage{amssymb}

\title{CS 2050: Discrete Mathematics for Computer Science}
\author{Jaeheon Shim}
\date{}

\begin{document}

\pagestyle{fancy}
\fancyhead[L]{CS 2050: Discrete Mathematics for Computer Science}

\maketitle
\thispagestyle{fancy}

\section{Proofs}
\subsection{Axioms}
An argument in a proof is either an axiom or rests on an axiom. An axiom is an unproven statement, and different theorems can becom true or false depending on your choice of axioms.\\\\
The following axioms/assumptions are allowed
\begin{itemize}
	\item The rules of algebra \\
		e.g. if x, y, z are real numbers and $x = y$, then $x + z = y + z$
	\item The set of integers is closed under addition, multiplication, and subtraction
	\item Every integer is either even or odd
	\item If x is an integer, there is no integer between x and x + 1
	\item The relative order of any two real numbers \\
		e.g. $\frac{1}{2} < 1$ or $4.2 \geq 3.7$
	\item The square of any number is greater than or equal to 0
\end{itemize}
\subsection{Existential Instantiation}
A law of logic that says if an object is known to exist, that object can be given a name as long as the name is not currently being used to denote something else. \\\\
\textbf{Example}\\
If $n$ is an odd integer, $n = 2k + 1$ for some integer $k$.
\subsection{Direct Proofs}
In a direct proof of a conditional statement $p \rightarrow c$, the hypothesis $p$ is assumed to be true and the conclusion $c$ is proven as a direct result of the assumption.
\begin{center}
	If n is an odd integer, then $n^2$ is an odd integer.
\end{center}
Many theorems also have a universal quantifier such as
\begin{center}
	For every integer n, if n is odd then $n^2$ is odd.
\end{center}

\subsubsection{Example}
\textbf{Theorem} \quad The square of every odd integer is also odd\\
\begin{proof}
	Let $n$ be an odd integer.\\
	Since $n$ is odd, $n = 2k + 1$ for some integer k. \\
	Plug $n = 2 + 1$ into $n^2$ to get: \\
	\begin{align}
		n^2 &= (2k + 1)^2 \\
			&= 4k^2 + 4k + 1 \\
			&= 2(2^2 + 2k) + 1
	\end{align}
	Since k is an integer, $2^2 + 2$ is also an integer. \\
	Since $n^2 = 2m + 1$, where $m = 2k^2 + 2$ is an integer, $n^2$ is odd.
\end{proof}

\textbf{Two-Column Proof Format} \\
\begin{tabular}{c|c}
	n is odd & Assume p. \\
	$n = 2k + 1$, for some $k \in \mathbf{Z}$ & Definition of Odd \\
	$n^2 = (2k + 1)^2$ & Square both sides \\
	$n^2 = 4k^2 + 4k + 1$ & expand $(2k+1)^2$ \\
	$n^2 = 2(2k^2 + 2k) + 1$ & factor out 2 \\
	$w = 2k^2 + 2m$ & define new variable \\
	$w$ is an integer & integers are closed under addition, multiplication, exponentiation \\
	$n^2 + 2w + 1$, w is integer & substitute w \\
	$n^2$ is odd & definition of odd \\
\end{tabular}
$\therefore$ Therefore, by direct proof, I have shown $p \rightarrow q$
\subsubsection{Example}
Prove that, if x and y are squares, then xy is a square \\\\
p: x and y are squares\\
q: xy is a square \\
prove $p \rightarrow q$
\begin{align*}
	\text{x and y are square integers} && \text{assume p} \\
	x = k^2, k \in \mathbf{Z} && \text{definition of square} \\
	y = j^2, j \in \mathbf{Z} && \text{definition of square} \\
	xy = k^2n^2 && \text{multiply two equations} \\
	xy = g^2, g \in \mathbf{Z}
\end{align*}
Therefore, by direct proof, I have shown $p \rightarrow q$
\subsection{Proof by Contrapositive}
To prove $p \rightarrow q$, we instead prove $\lnot q \rightarrow \lnot p$
\subsubsection{Example}
Prove that, for positive integers n, a, b, if $n = ab$, then $a \leq \sqrt{n}$ or $b \leq \sqrt{n}$
\begin{align*}
	\lnot[(a \leq \sqrt{n}) \text{or} (b \leq \sqrt{n})] \rightarrow \lnot(n = ab) \\
	a > \sqrt{n} \text{ and } b > \sqrt{n} && \text{assume } \lnot q \\
	a \cdot b > \sqrt{n}\sqrt{n} && \text{multiply inequalities}\\
	ab > n && \text{algebra} \\
	ab \neq n && \text{definition of >}
\end{align*}
Therefore, we have proven the contrapositive, so the original statement is true.
\subsection{Proof by Contradiction}
To prove \textbf{R}
\begin{enumerate}
	\item Assume you are wrong (\textbf{R} is false)
	\item Show that eventually, \textbf{R} being false leads to some absurdity (e.g. that $1 = 2$, or that p and not p are both simultaneously, etc.)
\end{enumerate}
\subsection{Example}
Prove $\sqrt{2}$ is irrational. \\\\
\textbf{Proof by Contradiction} \\
\begin{align*}
	\sqrt{2} \text{ is rational} && \text{assume negation of original statement} \\
	\sqrt{2} = \frac{a}{b}, a \in \mathbf{Z}, b \in \mathbf{Z}, b \neq 0 && \text{definition of rational} \\
																		 && \text{Let } \frac{a}{b} \text{ be the most simplified fraction for } \sqrt{2} \\ 
	2 = \left(\frac{a}{b}\right)^2 && \text{square both sides} \\
	2 = \frac{a^2}{b^2} \\
	2b^2 = a^2 && \text{multiply both sides by }b^2 \\
	a^2 \text{ is even} && \text{by the definition of even} \\
	a \text{ is even} && \text{proven previously} \\
	a = 2k, k \in \mathbf{Z} && \text{definition of even} \\
	a^2 = 4k^2 && \text{square both sides} \\
	2b^2 = 4k^2 && \text{substitute} \\
	b^2 = 2k^2 && \text{divide both sides by 2} \\
	k^2 \text{ is an integer} && \text{closure of } \mathbf{Z} \\
	b^2 \text{ is even} && \text{definition of even} \\
	b \text{ is even} && \text{proven previously}
\end{align*}
If $a$ and $b$ are not both even, then $\frac{a}{b}$ is not the most simplified form, therefore we have a contradiction. So our original assumption, that $\sqrt{2}$ is rational, is wrong. It must be the case that $\sqrt{2}$ is irrational.
\subsection{Proof by Cases}
Split the domain into cases and prove each case. The cases must cover the entire domain.
\section{Sets}
A set is an unordered collection of distinct objects. Below is an excerpt from my MATH 3012 notes regarding sets.
\subsection{Set Basics}

A set is a collection of objects. Below are two examples of sets:
\begin{align*}
	A &= \{1, 2, 3\} \\
	B &= \{A, -2.7, ARTICHOKE\}
\end{align*}
\begin{itemize}
	\item The order of elements does not matter
	\item Repeated elements are not allowed
\end{itemize}
The $\emptyset$ symbol denotes the empty set. The empty set is the set containing no elements.\\\\
Below are some symbols related to sets.
\begin{itemize}
	\item
		$\mathbf{b \in A}$ - element $b$ is a member of set $A$ \\
		$$1 \in \{1, 2, 3\}$$
		$$7 \not\in \{1, 2, 3\}$$
		$$\emptyset \in \{1, 2, 3\}$$
	\item
		$\mathbf{A \subseteq B}$ - A is a subset of B (Every element of A is also in B)
	\item
		$\mathbf{A \not\subseteq B}$ - A is not a subset of B (There exists an element of A that is not in B
	\item
		$\mathbf{\exists}$ - "there exists" \\
		$$\exists x \in \{7, 8\} \text{ such that } x > 5$$
	\item
		$\mathbf{\forall}$ - "for all" \\
		$$\forall x \in \{7, 8, 9\}, x \geq 4$$
	\item
		$\mathbf{B = \{x \in A \mid x > 4\}}$ - An example of set builder notation. "The set containing every element in A that is greater than 4"
	\item
		$\mathbf{\cup}$ - Set Union: Elements in either set or both\\
		$$ \{1, 2, 3, 4\} \cup \{3, 4, 5, 6\} = \{1, 2, 3, 4, 5, 6\} $$
	\item
		$\mathbf{\cap}$ - Set Intersection: Elements in both sets\\
		$$ \{1, 2, 3, 4\} \cap \{3, 4, 5, 6\} = \{3, 4\} $$
	\item
		$\mathbf{|A|}$ - Cardinality: The size of a set\\
		$$ A = \{1, 2, 3, 4\} \qquad |A| = 4$$
\end{itemize}

	\subsection{The Cartesian Product}
	The cartesian product is an operation that can be applied to two sets.\\\\
	\textbf{Example}\\
	Let $A = \{1, 2\},\quad B = \{3, 4\}$
	\begin{align*}
		A \times B &= \{(x, y) \mid x \in A, y \in B\} \\
		&= \{(1, 3), (1, 4), (2, 3), (2, 4)\} 
	\end{align*}

	More formally,
	$$ A_1 \times A_2 \times \cdots \times A_n = \{(a_1, a_2, \ldots, a_n) \mid a_i \in A_i \forall i \in {1, 2, \ldots, n} \} $$

	Notice the elements of the set produced by the cartesian product are contained in parentheses '()' instead of curly brackets '\{\}'. This denotes that they are a tuple. In a tiple, \textbf{order matters} and \textbf{repeates are allowed}

	\subsection{Set Proofs}
	\subsubsection{Proving $x \in A \rightarrow x \in B$}
	Can be proven by conventional methods (direct proof, contrapositive, etc.)
	\subsubsection{Proving $A \subset B$}
	\begin{enumerate}
		\item Prove $A \subseteq B$
		\item Prove $A \neq B$
	\end{enumerate}
	\textbf{Disproving}\\
	Either show $A \nsubseteq B$ or $A = B$. There is an element in A that is not in B, or vice versa.
	\subsubsection{Proving $A = B$}
	You need to prove that $A \subseteq B$ AND $B \subseteq A$
	\begin{enumerate}
		\item Prove $x \in A \rightarrow x \in B$ ($A \subseteq B$)
		\item Prove $x \in B \rightarrow x \in A$ ($B \subseteq A$)
	\end{enumerate}
	\subsubsection{Proof of transitivty of $\subset$}
	It is true that $A \subset B, B \subset D \rightarrow A \subset D$ \\
	\textbf{Direct Proof}\\
	Assume $A \subset B, B \subset D$
	\begin{enumerate}
		\item $x \in A \rightarrow x \in D \quad (A \subset D) $
			\begin{enumerate}
				\item $x \in A$ \qquad assumption
				\item $x \in A \rightarrow x \in B$ \qquad (defn of $A \subset B$)
				\item $x \in B$ \qquad modus ponens
				\item $x \in B \rightarrow x \in D$ (defn of $B \subset D$)
				\item $x \in D$ \qquad modus ponens
			\end{enumerate}
		\item $A \neq D$ \\
			We assume $B \subset D$. Let $q$ be something in D but not in B. Can q be in A? No, because $A \subseteq B$. (So if it was in A it would also HAVE to be in B).
	\end{enumerate}
	\section{Functions}
	$f: A \rightarrow B$ \qquad f is a function from A to B \\
	$f(a) = b$ \qquad f assigns b to a \\
	$A$ is the domain of f \\
	$B$ is the codomain (the range is not the codomain) \\\\
	\textbf{To be a well defined function, each preimage must be assigned exactly one image}
	\begin{itemize}
		\item preimage: a in $f(a) = b$
		\item image: b in $f(a) = b$
	\end{itemize}
	Is square root a well defined function?
	$$\sqrt{25} = 5 \qquad \sqrt{25} = -5$$
	It depends on what domain and codomain you give it. We can disallow negative outputs (principal square root), in which case the square root function is well defined.
	\subsection{Combining functions}
	If $f$ and $g$ are functions with the same domain and codomain, we can define things like
	$$f(x) + g(x) \qquad f(x) \cdot g(x) \qquad f(x) \circ g(x) = f(g(x))$$
	\subsection{Properties of functions}
	There are 3 properties of functions we would like to define
	\begin{itemize}
		\item One-to-One (Injection) \\
			A function is one-to-one if "different inputs have different outputs". $\forall a \forall b(f(a) = f(b) \rightarrow a = b)$. Multiple different inputs do not map to the same input. \\\\
			Is $x \mapsto x^2$ one-to-one? It depends on the domain. For $\mathbb{Z}$, no (e.g. 7 and -7 both map to 49). For $\mathbb{N}$, yes. \\\\
			\textbf{Examples of functions}
			\begin{itemize}
				\item $x \mapsto x^0$ Not one-to-one: everything maps to 1
				\item $x \mapsto -x$ One-to-one: No way for two different real numbers to map to same negative
				\item $x \mapsto |x|$ Not one-to-one: e.g. 5 and -5 both map to 5
				\item $x \mapsto \frac{1}{x}$ One-to-one
				\item $S \mapsto P(S)$ One-to-one
				\item $S \mapsto |S|$ Not one-to-one (two sets can have the same size without having the same elements
				\item $S \mapsto S^c$ One-to-one
			\end{itemize}
			
	\end{itemize}
	
\end{document}
