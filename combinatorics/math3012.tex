\documentclass[12pt]{article}
\usepackage[margin=0.75in,tmargin=1in]{geometry}
\usepackage{fancyhdr}
\usepackage{amsmath}

\title{MATH 3012: Applied Combinatorics}

\begin{document}
\pagestyle{fancy}
\fancyhead[L]{MATH 3012: Applied Combinatorics}

\section{Sets}
The first order of business is a review of basic set principles and notation.
\subsection{Sets}

A set is a collection of objects. Below are two examples of sets:
\begin{align*}
	A &= \{1, 2, 3\} \\
	B &= \{A, -2.7, ARTICHOKE\}
\end{align*}
\begin{itemize}
	\item The order of elements does not matter
	\item Repeated elements are not allowed
\end{itemize}
The $\emptyset$ symbol denotes the empty set. The empty set is the set containing no elements.\\\\
Below are some symbols related to sets.
\begin{itemize}
	\item
		$\mathbf{b \in A}$ - element $b$ is a member of set $A$ \\
		$$1 \in \{1, 2, 3\}$$
		$$7 \not\in \{1, 2, 3\}$$
		$$\emptyset \in \{1, 2, 3\}$$
	\item
		$\mathbf{A \subseteq B}$ - A is a subset of B (Every element of A is also in B)
	\item
		$\mathbf{A \not\subseteq B}$ - A is not a subset of B (There exists an element of A that is not in B
	\item
		$\mathbf{\exists}$ - "there exists" \\
		$$\exists x \in \{7, 8\} \text{ such that } x > 5$$
	\item
		$\mathbf{\forall}$ - "for all" \\
		$$\forall x \in \{7, 8, 9\}, x \geq 4$$
	\item
		$\mathbf{B = \{x \in A \mid x > 4\}}$ - An example of set builder notation. "The set containing every element in A that is greater than 4"
	\item
		$\mathbf{\cup}$ - Set Union: Elements in either set or both\\
		$$ \{1, 2, 3, 4\} \cup \{3, 4, 5, 6\} = \{1, 2, 3, 4, 5, 6\} $$
	\item
		$\mathbf{\cap}$ - Set Intersection: Elements in both sets\\
		$$ \{1, 2, 3, 4\} \cap \{3, 4, 5, 6\} = \{3, 4\} $$
	\item
		$\mathbf{|A|}$ - Cardinality: The size of a set\\
		$$ A = \{1, 2, 3, 4\} \qquad |A| = 4$$
\end{itemize}

	\subsection{The Cartesian Product}
	The cartesian product is an operation that can be applied to two sets.\\\\
	\textbf{Example}\\
	Let $A = \{1, 2\},\quad B = \{3, 4\}$
	\begin{align*}
		A \times B &= \{(x, y) \mid x \in A, y \in B\} \\
		&= \{(1, 3), (1, 4), (2, 3), (2, 4)\} 
	\end{align*}

	More formally,
	$$ A_1 \times A_2 \times \cdots \times A_n = \{(a_1, a_2, \ldots, a_n) \mid a_i \in A_i \forall i \in {1, 2, \ldots, n} \} $$

	Notice the elements of the set produced by the cartesian product are contained in parentheses '()' instead of curly brackets '\{\}'. This denotes that they are a tuple. In a tiple, \textbf{order matters} and \textbf{repeates are allowed}.\\\\
	\textbf{The Product Rule} \\ 
	$$|A \times B| = |A| \cdot |B|$$

	\section{Counting}
	\subsection{Fundamental Principles of Counting}
	\subsubsection{Rule of Sum}
	If there are $m$ possible choices for a task, and $n$ possible choices for another task, performing either task can be completed in one of $m + n$ ways. \\\\
	\textbf{Example}\\
	You can choose either sprinkles or syrup \textbf{but not both} as topping for your ice cream. There are 4 kinds of sprinkles and 3 kinds of syrup. Therefore, there are $4 + 3 = 7$ total different ways to choose a topping for your ice cream.
	\subsubsection{Rule of Product}
	If a choice has two stages with $m$ possibilities in the first stage and $n$ possibilities in the second stage, there are $(m)(n)$ total possibilities.\\\\
	\textbf{Example}\\
	You are at a restauraunt in Paris about to enjoy a three course meal. There are 3 choices for the appetizer, 4 choices for the main course, and 2 choices for dessert. Therefore, there are $(3)(4)(2) = 24$ different three course meals you are able to enjoy. 
	\subsection{Permutations}
	Permutations refer to the counting of possible rearrangements of some number of objects, assuming each object is distinguishable. \\\\
	Suppose we have 10 differently colored blocks, and we want to find out how many ordered arrangements of 4 blocks we can form.\\\\
	We can think of choosing each block in stages using the rule of product. We have 10 choices for the first block, 9 choices for the second, 8 for the third, and 7 for the fourth. Therefore, there are $10 * 9 * 8 * 7 = 5040$ possibilities.
	\begin{align*}
		10 * 9 * 8 * 7 &= \frac{10 * 9 * 8 * 7 * 6 * 5 * 4 * 3 * 2 * 1}{6 * 5 * 4 * 3 * 2 * 1} \\
		&= \frac{10!}{6!} \\
		&= \frac{10!}{(10 - 4)!}
	\end{align*}
	From this, we can derive a general formula for permutations. $n$ refers to the number of total objects, and $r$ refers to the number of objects we want to form a permutation with.
	$$P(n, r) = \frac{n!}{(n - r)!}$$
	\textbf{Example}\\
	How many ways are there to rearrange the word DATABASES into distinct strings? \\\\
	We can apply our logic of having 9 choices for the first letter, 8 for the second, ... However, this will \textit{overcount} certain strings, as there exist duplicates letters in the string. Therefore, we need to divide to correct for overcounting. \\\\
	There are 2 S's in DATABASES, the position of which can be flipped in the rearranged string. Therefore, we must divide by 2. Similarly, there are 3 A's, of which there are $3! = 6$ rearrangements. The number of distinct rearrangements we can create is
	$$\frac{9!}{3!2!} = 30240$$
	\subsection{Strings}
	Let $X$ be a set. A string over the \textit{alphabet} $X$ is a sequence of elements of X (order matters).\\\\
	\textbf{Example}\\
	Consider a set Y
	$$ Y = \{a, b, \ldots, z\} $$
	The following are examples of strings over Y
	\begin{itemize}
		\item (a, b, c, d, e)
		\item (h, e, l, l, o)
	\end{itemize}
	\textbf{Example}\\
	How many strings of length 6 are there over the alphabet $L = \{a, b, \ldots, z\}$ \textbf{with no repeated characters}? \\\\
	26 options for postion 1, 25 for position 2, 24 for position 3, ...
	$$(26)(25)(24)(23)(22)(21)$$
	More generally, let X be a set with $|X| = n$ and let $k \in \mathbf{N}$ with $k \leq n$ \\
	The number of strings of length k over X with no repeated characters is
	\begin{align*}
		(n)(n - 1)(n - 2)\cdots(n - k + 1) = \frac{n!}{(n - k)!}
	\end{align*}
	\subsection{Combinations}
	Suppose we have 10 differently colored blocks, and we want to find out how many different sets of 4 blocks we can choose from the set of 10 (order doesn't matter).\\\\
	We could start out by listing out the number of possible combinations.
	$$ \frac{10!}{(10 - 4)!} $$
	However, this formula would count each of the following arrangements of blocks (for example) even though they all contain the same 4 colors.
	\begin{align*}
		\{R, B,& O, G\}\\
		\{B, R,& G, O\}\\
		\{G, O,& R, B\}\\
		\{O, R,& B, G\}\\
			   &\vdots
	\end{align*}
	We need to correct for overcounting. Notice that there will be $4!$ different ordered arrangements of these same 4 colors. Let's update our formula.
	$$\frac{10!}{(10 - 4)!4!}$$
	We divide by $4!$ to account for overcounting, and our formula is now correct.\\\\
	More generally, there are $$\frac{n!}{(n - k)!k!}$$ ways of choosing a set of k elements from a set of n elements. There exists a special notation for this formula:
	$${n \choose k} = \frac{n!}{(n - k)! k!}$$
\end{document}
